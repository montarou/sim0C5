\documentclass[11pt,a4paper]{article}

% Packages
\usepackage[utf8]{inputenc}
\usepackage[T1]{fontenc}
\usepackage[french]{babel}
\usepackage{geometry}
\usepackage{graphicx}
\usepackage{booktabs}
\usepackage{siunitx}
\usepackage{amsmath}
\usepackage{amssymb}
\usepackage{xcolor}
\usepackage{float}
\usepackage{caption}
\usepackage{subcaption}
\usepackage{fancyhdr}
\usepackage{hyperref}
\usepackage{array}
\usepackage{multirow}

% Configuration
\geometry{margin=2.5cm}
\sisetup{
    separate-uncertainty=true,
    multi-part-units=single,
    per-mode=symbol
}

% Couleurs
\definecolor{darkblue}{RGB}{0,51,102}
\definecolor{lightgray}{RGB}{245,245,245}

% En-têtes
\pagestyle{fancy}
\fancyhf{}
\fancyhead[L]{\small Simulation Geant4 - MiniX}
\fancyhead[R]{\small \today}
\fancyfoot[C]{\thepage}

% Commandes personnalisées
\newcommand{\pGy}{\si{\pico\gray}}
\newcommand{\nGy}{\si{\nano\gray}}
\newcommand{\keV}{\si{\kilo\electronvolt}}

\begin{document}

%===============================================================================
% PAGE DE TITRE
%===============================================================================
\begin{titlepage}
    \centering
    \vspace*{2cm}
    
    {\Huge\bfseries Simulation Monte Carlo Geant4\\[0.5cm]
    Source X MiniX - Dosimétrie Eau}
    
    \vspace{1.5cm}
    
    {\Large\itshape Rapport de synthèse des résultats}
    
    \vspace{2cm}
    
    {\large
    \begin{tabular}{rl}
        \textbf{Nombre d'événements :} & 1\,000\,000 \\[0.3cm]
        \textbf{Code :} & Geant4 11.03-patch-01 \\[0.3cm]
        \textbf{Physics List :} & FTFP\_BERT \\[0.3cm]
        \textbf{Date :} & 30 janvier 2026 \\
    \end{tabular}
    }
    
    \vfill
    
    \begin{abstract}
    \noindent Ce rapport présente les résultats d'une simulation Monte Carlo de la source X MiniX 
    avec un million d'événements primaires. L'objectif est de caractériser la transmission 
    du faisceau à travers le collimateur et de quantifier la dose déposée dans les anneaux 
    d'eau concentriques du détecteur. Les résultats montrent une transmission de 1.36\% 
    après le collimateur et une dose totale de \SI{10.95}{\nano\gray} dans l'eau.
    \end{abstract}
    
\end{titlepage}

%===============================================================================
% TABLE DES MATIÈRES
%===============================================================================
\tableofcontents
\newpage

%===============================================================================
% SECTION 1: CONFIGURATION
%===============================================================================
\section{Configuration de la simulation}

\subsection{Géométrie}

La simulation modélise la source X MiniX avec les éléments suivants :
\begin{itemize}
    \item \textbf{Source :} Émission de photons X (spectre MiniX caractéristique)
    \item \textbf{Collimateur :} Aluminium + Laiton (ouverture \SI{4}{\milli\meter})
    \item \textbf{Plans de scoring :} 5 plans à différentes positions en $z$
    \item \textbf{Détecteur :} 5 anneaux d'eau concentriques (rayon 0--10~mm)
\end{itemize}

\subsection{Plans de mesure}

\begin{table}[H]
\centering
\caption{Configuration des plans de scoring}
\label{tab:plans}
\begin{tabular}{llc}
\toprule
\textbf{Ntuple} & \textbf{Position} & \textbf{Description} \\
\midrule
\texttt{plane\_passages} & $z = \SI{18}{\milli\meter}$ & Après collimateur \\
\texttt{ScorePlane2\_passages} & $z = \SI{28}{\milli\meter}$ & Plan intermédiaire 1 \\
\texttt{ScorePlane3\_passages} & $z = \SI{38}{\milli\meter}$ & Plan intermédiaire 2 \\
\texttt{WaterRings\_passages} & $z = 65$--$\SI{68}{\milli\meter}$ & Couronnes d'eau \\
\texttt{ScorePlane5\_passages} & $z = \SI{70}{\milli\meter}$ & Après conteneur \\
\bottomrule
\end{tabular}
\end{table}

\subsection{Anneaux d'eau}

Le détecteur est constitué de 5 anneaux cylindriques d'eau de \SI{3}{\milli\meter} d'épaisseur :

\begin{table}[H]
\centering
\caption{Configuration des anneaux d'eau}
\label{tab:anneaux}
\begin{tabular}{ccccc}
\toprule
\textbf{Anneau} & \textbf{$r_{min}$ (mm)} & \textbf{$r_{max}$ (mm)} & \textbf{Volume (\si{\milli\meter^3})} & \textbf{Masse (g)} \\
\midrule
0 & 0 & 2 & 37.70 & 0.0377 \\
1 & 2 & 4 & 113.10 & 0.1131 \\
2 & 4 & 6 & 188.50 & 0.1885 \\
3 & 6 & 8 & 263.89 & 0.2639 \\
4 & 8 & 10 & 339.29 & 0.3393 \\
\midrule
\textbf{Total} & 0 & 10 & 942.48 & 0.9425 \\
\bottomrule
\end{tabular}
\end{table}

%===============================================================================
% SECTION 2: TRANSMISSION
%===============================================================================
\section{Transmission du faisceau}

\subsection{Statistiques par plan}

La transmission des particules à travers les différents plans de scoring est résumée 
dans le tableau~\ref{tab:transmission}.

\begin{table}[H]
\centering
\caption{Transmission des particules (1M événements primaires)}
\label{tab:transmission}
\begin{tabular}{lrrrc}
\toprule
\textbf{Plan} & \textbf{Total} & \textbf{Primaires} & \textbf{Secondaires} & \textbf{Transmission (\%)} \\
\midrule
$z = \SI{18}{\milli\meter}$ (après collim.) & 13\,622 & 13\,613 & 9 & 1.36 \\
$z = \SI{28}{\milli\meter}$ & 13\,427 & 13\,426 & 1 & 1.34 \\
$z = \SI{38}{\milli\meter}$ & 13\,196 & 13\,194 & 2 & 1.32 \\
$z = 65$--$\SI{68}{\milli\meter}$ (eau) & 12\,560 & 12\,554 & 6 & 1.26 \\
$z = \SI{70}{\milli\meter}$ (après eau) & 2\,688 & 2\,686 & 2 & 0.27 \\
\bottomrule
\end{tabular}
\end{table}

\subsection{Taux de transmission}

Pour 10\,000 événements primaires :
\begin{itemize}
    \item Après collimateur : $136 \pm 14$ photons (1.36\%)
    \item Atteignant l'eau : $\sim 126$ photons
    \item Sortant de l'eau : $\sim 27$ photons
\end{itemize}

L'atténuation dans l'eau est de $\sim 79\%$, ce qui correspond à un coefficient 
d'atténuation effectif cohérent avec les photons de basse énergie (< 50 keV).

%===============================================================================
% SECTION 3: DOSIMÉTRIE
%===============================================================================
\section{Dosimétrie}

\subsection{Dose totale dans l'eau}

\begin{table}[H]
\centering
\caption{Énergie déposée et dose dans l'eau (run complet)}
\label{tab:dose_totale}
\begin{tabular}{lcc}
\toprule
\textbf{Paramètre} & \textbf{Valeur} & \textbf{Unité} \\
\midrule
Énergie totale déposée & 64\,130 & \keV \\
Dose totale (run) & 10\,901 & \pGy \\
Dose totale (run) & 10.90 & \nGy \\
\midrule
Dose par 10k événements (moyenne) & $109 \pm 15$ & \pGy \\
\bottomrule
\end{tabular}
\end{table}

\subsection{Dose par anneau}

\begin{table}[H]
\centering
\caption{Dose déposée par anneau (run complet -- 1M événements)}
\label{tab:dose_anneau}
\begin{tabular}{cccccc}
\toprule
\textbf{Anneau} & \textbf{Rayon (mm)} & \textbf{Énergie (keV)} & \textbf{Dose (pGy)} & \textbf{Dose (nGy)} & \textbf{Fraction (\%)} \\
\midrule
0 & 0--2 & 3\,627 & 15\,411 & 15.41 & 23.1 \\
1 & 2--4 & 11\,802 & 16\,717 & 16.72 & 25.1 \\
2 & 4--6 & 20\,382 & 17\,322 & 17.32 & 26.0 \\
3 & 6--8 & 26\,842 & 16\,295 & 16.30 & 24.5 \\
4 & 8--10 & 1\,478 & 698 & 0.70 & 1.0 \\
\midrule
\textbf{Total} & 0--10 & 64\,130 & 10\,901 & 10.90 & -- \\
\bottomrule
\end{tabular}
\end{table}

\subsection{Observations}

\begin{itemize}
    \item La dose est relativement uniforme dans les anneaux 0--3 (15--17 nGy)
    \item L'anneau 4 (périphérique) reçoit très peu de dose (0.7 nGy, soit 1\%)
    \item Le maximum de dose se situe dans l'anneau 2 (4--6 mm)
    \item Le profil de dose reflète la distribution spatiale du faisceau collimaté
\end{itemize}

\subsection{Distribution statistique}

Les histogrammes de dose par batch de 10\,000 événements montrent une distribution 
quasi-gaussienne :

\begin{table}[H]
\centering
\caption{Statistiques des distributions de dose (par 10k événements)}
\label{tab:dose_stats}
\begin{tabular}{lccc}
\toprule
\textbf{Zone} & \textbf{Moyenne (pGy)} & \textbf{Écart-type (pGy)} & \textbf{CV (\%)} \\
\midrule
Dose totale & 109 & 15 & 14 \\
Anneau 0 & 154 & 87 & 56 \\
Anneau 1 & 167 & 45 & 27 \\
Anneau 2 & 173 & 39 & 23 \\
Anneau 3 & 164 & 31 & 19 \\
Anneau 4 & 7 & 8 & 114 \\
\bottomrule
\end{tabular}
\end{table}

Le coefficient de variation (CV) élevé pour l'anneau 0 et 4 s'explique par le faible 
nombre de photons atteignant ces régions (centre et périphérie).

%===============================================================================
% SECTION 4: VÉRIFICATION
%===============================================================================
\section{Vérification et cohérence}

\subsection{Comparaison LOG vs ROOT}

Une vérification de cohérence a été effectuée entre les valeurs du fichier log 
et celles extraites du fichier ROOT :

\begin{table}[H]
\centering
\caption{Vérification de cohérence LOG vs ROOT}
\label{tab:verification}
\begin{tabular}{lccc}
\toprule
\textbf{Paramètre} & \textbf{LOG} & \textbf{ROOT} & \textbf{Écart} \\
\midrule
Dose totale (pGy) & 10\,900.6 & 10\,950.0 & 0.5\% \\
Entrées plane\_passages & 13\,622 & 13\,622 & 0 \\
\bottomrule
\end{tabular}
\end{table}

\textbf{Conclusion :} Les valeurs sont parfaitement cohérentes entre les deux sources.

\subsection{Vérification des histogrammes}

\begin{itemize}
    \item[$\checkmark$] Histogrammes de dose correctement remplis (99 entrées)
    \item[$\checkmark$] Plages adaptées en pGy (pas de débordement)
    \item[$\checkmark$] Colonne \texttt{is\_secondary} présente dans les ntuples
    \item[$\checkmark$] Structure harmonisée (10 colonnes par ntuple)
\end{itemize}

%===============================================================================
% SECTION 5: OPTIMISATIONS
%===============================================================================
\section{Optimisations du code}

\subsection{Réduction de la verbosité}

Le fichier log a été optimisé pour réduire sa taille tout en conservant les 
informations essentielles :

\begin{table}[H]
\centering
\caption{Réduction du fichier log}
\label{tab:log}
\begin{tabular}{lcc}
\toprule
\textbf{Métrique} & \textbf{Avant} & \textbf{Après} \\
\midrule
Nombre de lignes & $\sim$42\,000 & 1\,447 \\
Taille du fichier & 2.8 Mo & 94 Ko \\
\textbf{Réduction} & \multicolumn{2}{c}{\textbf{96.6\%}} \\
\bottomrule
\end{tabular}
\end{table}

\subsection{Messages conservés}

Les messages suivants ont été conservés pour le suivi de la simulation :
\begin{itemize}
    \item \texttt{[PROGRESS]} : Avancement tous les 10\,000 événements
    \item Résumé de fin de run (doses, compteurs, statistiques)
    \item Messages d'initialisation (géométrie, matériaux)
    \item Diagnostics limités (3--5 premiers événements)
\end{itemize}

%===============================================================================
% SECTION 6: CONCLUSIONS
%===============================================================================
\section{Conclusions}

Cette simulation de 1 million d'événements de la source MiniX a permis de :

\begin{enumerate}
    \item \textbf{Caractériser la transmission} : 1.36\% des photons traversent 
    le collimateur, avec une atténuation de 79\% dans l'eau.
    
    \item \textbf{Quantifier la dose} : La dose totale déposée dans l'eau est de 
    \SI{10.9}{\nano\gray} pour 1M d'événements, soit \SI{109}{\pico\gray} par 
    10\,000 photons primaires.
    
    \item \textbf{Établir le profil de dose} : La dose est relativement uniforme 
    dans les anneaux 0--3 (15--17 nGy), avec un maximum dans l'anneau 2 (4--6 mm).
    
    \item \textbf{Valider le code} : Les corrections apportées (structure des ntuples, 
    unités des histogrammes) ont été vérifiées et validées.
    
    \item \textbf{Optimiser le code} : La taille du fichier log a été réduite de 96.6\% 
    sans perte d'information utile.
\end{enumerate}

\subsection{Recommandations}

Pour les simulations futures :
\begin{itemize}
    \item Utiliser au minimum 1M événements pour des statistiques fiables
    \item Vérifier la cohérence LOG/ROOT après chaque modification du code
    \item Les unités de dose sont en pGy dans les histogrammes ROOT
\end{itemize}

%===============================================================================
% ANNEXES
%===============================================================================
\appendix
\section{Formules de conversion}

\subsection{Unités de dose}

\begin{align}
    \SI{1}{\micro\gray} &= \SI{1000}{\nano\gray} = \SI{e6}{\pico\gray} \\
    \SI{1}{\nano\gray} &= \SI{1000}{\pico\gray}
\end{align}

\subsection{Calcul de la dose}

La dose est calculée selon :
\begin{equation}
    D \, [\text{pGy}] = \frac{E_{\text{dep}} \, [\text{keV}] \times 0.1602}{m \, [\text{g}]}
\end{equation}

où $E_{\text{dep}}$ est l'énergie déposée et $m$ la masse de l'anneau.

Le facteur de conversion provient de :
\begin{equation}
    \SI{1}{\keV} = \SI{1.602e-16}{\joule} = \SI{1.602e-13}{\milli\joule}
\end{equation}

et $\SI{1}{\gray} = \SI{1}{\joule\per\kilogram}$.

\section{Structure des ntuples}

Après harmonisation, tous les ntuples ont la structure suivante (10 colonnes) :

\begin{table}[H]
\centering
\small
\begin{tabular}{clll}
\toprule
\textbf{Col.} & \textbf{Nom} & \textbf{Type} & \textbf{Description} \\
\midrule
0 & \texttt{pdg} & int & Code PDG (22 = $\gamma$, 11 = $e^-$) \\
1 & \texttt{name} & string & Nom de la particule \\
2 & \texttt{is\_secondary} & int & 0 = primaire, 1 = secondaire \\
3 & \texttt{x\_mm} & double & Position X (mm) \\
4 & \texttt{y\_mm} & double & Position Y (mm) \\
5 & \texttt{z\_mm} & double & Position Z (mm) \\
6 & \texttt{ekin\_keV} & double & Énergie cinétique (keV) \\
7 & \texttt{trackID} & int & Identifiant du track \\
8 & \texttt{parentID} & int & ID du parent (0 = primaire) \\
9 & \texttt{creator\_process} & string & Processus créateur \\
\bottomrule
\end{tabular}
\caption{Structure harmonisée des ntuples}
\label{tab:ntuple_structure}
\end{table}

\end{document}
